% -*-coding: utf-8 -*-

\def\usewhat{pdflatex}    % 你喜欢哪种编译方式,pdflatex dvipdfmx xelatex yap 
% 目前texlive2009和miktex2.8(ctex2.8基于miktex2.8)中均提供了ctex中文支持宏包,在使用pdflatex、dvipdfmx和xelatex时,
% 用户无需再进行中文配置,安装texlive09或miktex2.8后就可编译本模板。
% 对于windows用户,若使用xelatex编译方式,请根据XP及以前系统,还是Vista及以后系统,在setup/definition.tex文件中选择合适的字体,然后再编译。
% 由于dvipspdf编译方式相对其他编译方式比较麻烦,并且没有什么优势,texlive09中ctex宏包没有默认提供
% dvips的中文支持方式,本模板不再提供dvipspdf编译方式。

%定义xelatex的中间临时变量,若\usewhat为xelatex时,后面执行xelatx的相关选项
\def\atempxetex{xelatex} %这一项无需改动
%input "reference\reference.bib" %for winedt users
\def\version{1.9.4.20100419}         % 该变量仅用于模板文件的版本号控制,新的论文规范从1.9开始;
	% 自从版本1.9.4.20100419后开始提供基于texlive09,miktex2.8的TeX系统的支持,不再支持早期的tl08和miktex2.7及以前版本。
	% 请使用tl08和miktex2.7及以前版本的用户使用本模板的1.9.2.10090424版本。

\def \xuewei {Doctor}   % 定义学位 博士
%\def \xuewei {Master}    % 硕士

\def\oneortwoside{twoside} %定义单双面打印,只对硕士学位论文有效;
%\def\oneortwoside{oneside} % 硕士单面打印

\def\xueke{Engineering} % 定义学科 工学
%\def\xueke{Science}      % 理学
%\def\xueke{Management}   % 管理学
%\def\xueke{Arts}         % 艺术学

\input{setup/type.tex}    % 硕博类型

%下面的book选项中可以使用 draft 选项,使插入的图形只显示外框,以加快预览速度。
\documentclass[12pt,a4paper,openany,\oneortwoside]{book}
%\documentclass[12pt,a4paper,openright,\oneortwoside]{book}

\input{setup/package.tex} % 引用的宏包

% 论文包含的内容
% FIXME: I don't know why I have to put Resume last :( Anyway, the
% order here does not seem matter.
%\includeonly{
                %body/introduction,
                %body/conclusion
                %appendix/publications,
                %appendix/acknowledgements,
                %appendix/Authorization,
                %appendix/Resume
            %}
\graphicspath{{figures/}} %定义所有的eps文件在 figures 子目录下

\begin{document}
\ifx\atempxetex\usewhat\else
\begin{CJK*}{UTF8}{song}
\fi

\input{setup/Definition} % 文本格式定义
\input{setup/format}

%%%%%%%%%%%%%%%%%%%%%%%%%%%%%%%%%%%%%%%%%%%%%%%%%%%%%%%%%%%
% 正文部分
%%%%%%%%%%%%%%%%%%%%%%%%%%%%%%%%%%%%%%%%%%%%%%%%%%%%%%%%%%%
\frontmatter
\sloppy % 解决中英文混排的断行问题,会加入间距,但不会影响断行
% -*-coding: utf-8 -*-

\newcommand{\chinesethesistitle}{西北工业大学硕博士学位论文~\LaTeX~模板} %授权书用,无需断行
\newcommand{\englishthesistitle}{\LaTeX~Dissertation Template of Northwestern Polytechnical University} %\uppercase作用:将英文标题字母全部大写;
\newcommand{\chinesethesistime}{2011~年~11~月}  %封面底部的日期中文形式
\newcommand{\englishthesistime}{November 2011}    %封面底部的日期英文形式

\ctitle{西北工业大学硕博士学位论文\\~\LaTeX~模板}  %封面用论文标题,自己可手动断行
\cdegree{\cxueke\cxuewei}
\csubject{计算机科学与技术}                 %(~按二级学科填写~)
\caffil{西北工业大学} %(在校生填所在系名称,同等学力人员填工作单位)
\cauthor{某~~某~~某}
\csupervisor{某~~某~~某~~~~教~~授} %导师名字
%\cassosupervisor{某~~~~~~某~~~~教~~授}     %(~如无副导师可以不列此项~)
%\ccosupervisor{某~~某~~某~~~~教~~授~} %(~如无联合培养导师则不列此项~)
\cdate{\chinesethesistime}

\etitle{\englishthesistitle}
\edegree{\exuewei \ of \exueke}
\esubject{Computer Science and  Technology}  %英文二级学科名
\eaffil{School \hfill of\hfill Computer\hfill Science}%英文单位 %换行用\newline,不要用\\
\eauthor{Alice}                   %作者姓名 (英文)
\esupervisor{Prof. Lambda}       % 导师姓名 (英文)
%\ecosupervisor{Professor X}
%\eassosupervisor{Professor Y}
\edate{\englishthesistime}

\natclassifiedindex{TP393}  %国内图书分类号
\internatclassifiedindex{681.324}  %国际图书分类号
\statesecrets{公开} %秘密

\iffalse
\BiAppendixChapter{摘~~~~要}{}  %使用winedt编辑时文档结构图(toc)中为了显示摘要,故增加此句;
\fi
\cabstract{
该模板是以哈尔滨工业大学\LaTeX硕博士学位论文模板为基础(哈工大
PlutoThesis项目,UTF8版本,svn版本号为r132),根据西北工业大学学位论文
规范制作的\LaTeX硕博士论文模板。该模板使用pdflatex在Linux环境下
(Ubuntu904, textlive 2009)通过编译测试。

模板主要做了以下几方面的修改:
\begin{itemize}[,leftmargin=4em,itemindent=0em]
\renewcommand{\labelitemi}{{\large$\bullet$}}
\item 移除了英文图表和目录,修改了封面的内容和页眉的显示方式,添加了知
    识产权声明,修改了参考文献的格式。
    
\item 做了些小的调整,具体修改请diff r132。

\end{itemize} 
}

\ckeywords{\LaTeX,论文模板,西北工业大学}


\eabstract{
This is a \LaTeX dissertation template of Northwestern Polytechnical
University, which is built according to the required format. The
template is derived from PlutoThesis project.
}

\ekeywords{\LaTeX, dissertation template, Northwestern Polytechnical
University}

\makecover
\clearpage
 % 封面

%% 中英目录
\renewcommand{\baselinestretch}{1}
\fontsize{12pt}{12pt}\selectfont
\clearpage{\pagestyle{empty}\cleardoublepage}
\pdfbookmark[0]{目~~~~录}{mulu}
\tableofcontents    % 中文目录
\ifxueweidoctor     % 英文目录右开
  \clearpage{\pagestyle{empty}\cleardoublepage}
\else%
  \ifoneortwoside\clearpage{\pagestyle{empty}\cleardoublepage}\fi
\fi

%% -*-coding: utf-8 -*-

%% 中英文插图、表格、算法索引   
%% 硕博士学位论文规范均不要求这一项,请正式打印的时候在main.tex中屏蔽掉% -*-coding: utf-8 -*-

%% 中英文插图、表格、算法索引   
%% 硕博士学位论文规范均不要求这一项,请正式打印的时候在main.tex中屏蔽掉% -*-coding: utf-8 -*-

%% 中英文插图、表格、算法索引   
%% 硕博士学位论文规范均不要求这一项,请正式打印的时候在main.tex中屏蔽掉\input{figtab.tex};

\ifxueweidoctor
  %\clearpage{\pagestyle{empty}\cleardoublepage}   % 清除目录后面空页的页眉和页脚
\else%
  {\ifoneortwoside\clearpage{\pagestyle{empty}\cleardoublepage}\else\newpage\fi} % 清除目录后面空页的页眉和页脚
\fi

\addcontentsline{toc}{chapter}{\hei 插~~~~图}   % 中文插图加入到中文目录
\listoffigures                                  % 生成中文 图形索引

\ifxueweidoctor
  %\clearpage{\pagestyle{empty}\cleardoublepage}   % 清除目录后面空页的页眉和页脚
\else%
  {\ifoneortwoside\clearpage{\pagestyle{empty}\cleardoublepage}\else\newpage\fi}   % 清除目录后面空页的页眉和页脚
\fi

\addcontentsline{toc}{chapter}{\hei 表~~~~格}   % 中文表格加入到中文目录
\listoftables                                   % 生成中文 表格索引


\ifxueweidoctor
  %\clearpage{\pagestyle{empty}\cleardoublepage}   % 清除目录后面空页的页眉和页脚
\else%
  {\ifoneortwoside\clearpage{\pagestyle{empty}\cleardoublepage}\else\newpage\fi}   % 清除目录后面空页的页眉和页脚
\fi

%%% Local Variables: 
%%% mode: latex
%%% TeX-master: "../main"
%%% End: 
;

\ifxueweidoctor
  %\clearpage{\pagestyle{empty}\cleardoublepage}   % 清除目录后面空页的页眉和页脚
\else%
  {\ifoneortwoside\clearpage{\pagestyle{empty}\cleardoublepage}\else\newpage\fi} % 清除目录后面空页的页眉和页脚
\fi

\addcontentsline{toc}{chapter}{\hei 插~~~~图}   % 中文插图加入到中文目录
\listoffigures                                  % 生成中文 图形索引

\ifxueweidoctor
  %\clearpage{\pagestyle{empty}\cleardoublepage}   % 清除目录后面空页的页眉和页脚
\else%
  {\ifoneortwoside\clearpage{\pagestyle{empty}\cleardoublepage}\else\newpage\fi}   % 清除目录后面空页的页眉和页脚
\fi

\addcontentsline{toc}{chapter}{\hei 表~~~~格}   % 中文表格加入到中文目录
\listoftables                                   % 生成中文 表格索引


\ifxueweidoctor
  %\clearpage{\pagestyle{empty}\cleardoublepage}   % 清除目录后面空页的页眉和页脚
\else%
  {\ifoneortwoside\clearpage{\pagestyle{empty}\cleardoublepage}\else\newpage\fi}   % 清除目录后面空页的页眉和页脚
\fi

%%% Local Variables: 
%%% mode: latex
%%% TeX-master: "../main"
%%% End: 
;

\ifxueweidoctor
  %\clearpage{\pagestyle{empty}\cleardoublepage}   % 清除目录后面空页的页眉和页脚
\else%
  {\ifoneortwoside\clearpage{\pagestyle{empty}\cleardoublepage}\else\newpage\fi} % 清除目录后面空页的页眉和页脚
\fi

\addcontentsline{toc}{chapter}{\hei 插~~~~图}   % 中文插图加入到中文目录
\listoffigures                                  % 生成中文 图形索引

\ifxueweidoctor
  %\clearpage{\pagestyle{empty}\cleardoublepage}   % 清除目录后面空页的页眉和页脚
\else%
  {\ifoneortwoside\clearpage{\pagestyle{empty}\cleardoublepage}\else\newpage\fi}   % 清除目录后面空页的页眉和页脚
\fi

\addcontentsline{toc}{chapter}{\hei 表~~~~格}   % 中文表格加入到中文目录
\listoftables                                   % 生成中文 表格索引


\ifxueweidoctor
  %\clearpage{\pagestyle{empty}\cleardoublepage}   % 清除目录后面空页的页眉和页脚
\else%
  {\ifoneortwoside\clearpage{\pagestyle{empty}\cleardoublepage}\else\newpage\fi}   % 清除目录后面空页的页眉和页脚
\fi

%%% Local Variables: 
%%% mode: latex
%%% TeX-master: "../main"
%%% End: 
  %图表索引, 如果不需要图表索引,注释掉这一句即可;
% \notation  %主要符号表
\addtocontents{toc}{\protect\vskip1\baselineskip} % 中文目录增加空行
\addtocontents{toe}{\protect\vskip1\baselineskip} % 英文目录增加空行

\ifxueweidoctor
  \clearpage{\pagestyle{empty}\cleardoublepage}   % 清除目录后面空页的页眉和页脚
\else%
  \ifoneortwoside\clearpage{\pagestyle{empty}\cleardoublepage}\fi  % 清除目录后面空页的页眉和页脚
\fi                                               %  第一章是否右开

\mainmatter
\defaultfont % 对应于小四的标准字号为12pt, 可以在正文中用此命令修改所需要字体的的大小

% -*-coding: utf-8 -*-

\defaultfont

\BiChapter{ 绪~~~论}{ }

\BiSection{背景 }{ }

虽然微软Office所见即所得的方式很容易上手,但当用于较为正式的写作时还是
会让作者花费大量时间在论文本身之外的排版工作。\LaTeX可靠而稳定的排版系
统可以让作者将精力集中于论文写作本身。如果你开始尝试使用\LaTeX进行论文
或者其他较为正式的文稿写作,你会慢慢爱上她。

参考文献对应的BibTeX文件位于\texttt{reference/reference.bib},这里推荐
使用JabRef\footnote{\url{http://jabref.sourceforge.net/}}对参考文献进
行管理。中英文摘要位于\texttt{preface}目录下的\texttt{cover.tex}文件中
。\texttt{appendix}目录中包含了致谢、论文发表情况、科研经历和知识产权
声明。\texttt{body}目录中包含了论文的主要章节。\texttt{setup}目录则是
一些相关的设置。


表~\ref{table:intro-motes} 给出了一个表的例子。图片一般放在
\texttt{figures}目录下,图~\ref{fig:intro-npu} 给出了一个图片的例子。
文献~\cite{lukaicheng2002} 是一本书,文献~\cite{wsn:jos} 是一篇期刊的
论文。

\begin{table}[htbp]
\TableBiCaption{ 几种常见传感器节点硬件基本参数 } {}
\vspace{2mm}
\label{table:intro-motes}
\centering
\begin{tabular}{ l l l l l l }
\toprule
\bfseries            &\bfseries              &\bfseries Data       &\bfseries Program    &\bfseries External   &\bfseries Data\\
\bfseries            &\bfseries Processor    &\bfseries memory     &\bfseries memory     &\bfseries flash      &\bfseries rate   \\
\bfseries Platforms  &\bfseries (MHz / bits) &\bfseries (KB)       &\bfseries (KB)       &\bfseries (KB)       &\bfseries (kbps) \\
\midrule
              Mica2  & 7.4 / 8      & 4          & 128    & 512    & 38 \\
              MicaZ  & 7.4 / 8      & 4          & 128    & 512    & 250 \\
              TelosB & 4.0 / 16     & 10         &  48    & 1024   & 250 \\
              Iris   & 12.0 / 8     & 8          & 128    & 512    & 250 \\
\bottomrule
\end{tabular}
\end{table}

\begin{figure}[htbp]
\centering
\includegraphics[width = 0.9\textwidth]{npu}
\FigureBiCaption{西北工业大学长安校区}{}
\label{fig:intro-npu}
\end{figure}

\BiSection{其他说明}{ }

论文使用pdflatex编译,在Linux系统下如果需要使用Windows下的字体,可能需
需要制作字体,具体方法请上网Google。另外,也可尝试对字体支持更好的
xetex编译,但可能需要进行一些修改。

main.tex中有每一章从右侧/奇数页开始的选项(openleft),但插入的并不是一
个完全的空白页。如需空白页,可以关闭openleft选项,然后手动在合适位置使
用如下命令插入空白页:
\begin{verbatim}
\newpage
~~~\vspace{1em}
\thispagestyle{empty}
\end{verbatim}

编译时直接使用\texttt{make}命令,得到\texttt{main\_pdflatex.pdf}文件。


% -*-coding: utf-8 -*-

\defaultfont

\BiChapter{ 总结与展望 }{ }

\newpage
~~~\vspace{1em}
\thispagestyle{empty}




%参考文献
\defaultfont
\ifx\atempxetex\usewhat
\bibliographystyle{chinesebst2005_UTF8}
\else
\bibliographystyle{chinesebst2005_UTF8}
\fi
\addcontentsline{toc}{chapter}{\hei \ReferenceCName}      % 参考文献加入到中文目录
\addcontentsline{toe}{chapter}{\bfseries \xiaosi \ReferenceEName} % 参考文献加入到英文目录
\addtolength{\bibsep}{-0.8 em} \nocite{*}
\bibliography{reference/reference}

\ifoneortwoside
\newpage
~~~\vspace{1em}
\thispagestyle{empty}
\fi

%\addtocontents{fen}{\protect\vskip1\baselineskip}
%\addtocontents{ten}{\protect\vskip1\baselineskip}
%英文图形和表格索引里加入空白行,通常放在 \include{appendix/appA}% 附录A之前。
%区分开正文和附录的图形和表格,一般没有这个必要。

% -*-coding: utf-8 -*-

\defaultfont
\BiAppendixChapter{攻读\cxuewei 学位期间发表的学术论文} {Papers
Published in the Period of Ph. D. Education}

% 参见...\Accessories\ThesisCriterion\参考文献国标GB7714-2005.pdf内规定
(一)国际会议论文
\begin{publist}
\item \underline{San Zhang}, Si Li, Er Wang. title, \textit{conference
    name}, publisher, city, Country, year. (\textbf{EI Compendex,
    Access Number: xxxxxxxxxxxxxx})

\end{publist}

(二)期刊论文
\begin{publist}
\item \underline{张三}, 李四. 题目, 期刊, 年, 卷(期): 页码
\end{publist}
 

%(二)申请及已获得的专利(无专利时此项可以不列出)


%(三)获得的科技奖励(无获奖时此项不必列出)

% 注:
% 如已发表的学术论文被EI或SCI收录,请标明收录号及SCI论文的影响因子;
% 对已接收但尚未发表出来的学术论文,请注明是否EI或SCI刊源。

\ifoneortwoside
\newpage
~~~\vspace{1em}
\thispagestyle{empty}
\fi



    % 所发文章
% -*-coding: utf-8 -*-

\defaultfont

\BiAppendixChapter{攻读\cxuewei 学位期间科研工作和获奖情况}{Resume}

{\hei 科研工作}
\begin{publist}
\item “项目名称”,国家自然科学基金课题 (项目编号)
\item “项目名称”,国家科技支撑计划 (项目编号)
\end{publist}

%{\hei 获奖情况}
%\begin{publist}
%\item xxxx~年~x~月~~~xxxxxxx
%\end{publist}

\vspace{1cm}
{\hei 专利情况}
\begin{publist}
\item 张三, 李四, \underline{王二}. 专利名称. 发明专利 (专利号)
\end{publist}

\ifoneortwoside
\newpage
~~~\vspace{1em}
\thispagestyle{empty}
\fi


          % 个人简历
% -*-coding: utf-8 -*-

\defaultfont

\BiAppendixChapter{致~~~~谢}{Acknowledgement}

\ifoneortwoside
\newpage
~~~\vspace{1em}
\thispagestyle{empty}
\fi



% 致谢
% -*-coding: utf-8 -*-

\newpage
\thispagestyle{empty}

\begin{center}
    {\xiaoer {\hei 西北工业大学 \\ 学位论文知识产权声明书}}
\end{center}

本人完全了解学校有关保护知识产权的规定,即:研究生在校攻读学位期间论文
工作的知识产权单位属于西北工业大学。学校有权保留并向国家有关部门或机构
送交论文的复印件和电子版。本人允许论文被查阅和借阅。学校可以将本学位论
文的全部或部分内容编入有关数据库进行检索,可以采用影印、缩印或扫描等复
制手段保存和汇编本学位论文。同时本人保证,毕业后结合学位论文研究课题再
撰写的文章一律注明作者单位为西北工业大学。

保密论文待解密后适用本声明。

\begin{flushleft}
~~~~~~~~学位论文作者签名:~~~~~~~~~~~~~~~~~~~~~~~~~~~~~指导教师签名:~~~~~~~~~~~~~~~~~~~~~~~~~~~~~~~~~~~~~~~~~

~~~~~~~~~~~~~~~~~~~~~~~~~~~~~~~~~~~~~~~~~年~~~~~~月~~~~~~日~~~~~~~~~~~~~~~~~~~~~~~~~~~~~~~~~年~~~~~~月~~~~~~日~~~~
\end{flushleft}

\vspace{1cm}
\begin{flushleft}
----------------------------------------------------------------------------------------------------------
\end{flushleft}
\vspace{1cm}

\begin{center}
    {\xiaoer {\hei 西北工业大学 \\ 学位论文原创性声明}}
\end{center}

秉承学校严谨的学风和优良的科学道德,本人郑重声明:所呈交的学位论文,是本
人在导师的指导下进行研究工作所取得的成果。尽我所知,除文中已经注明引用
的内容和致谢的地方外,本论文不包含任何其他个人或集体已经公开发表或撰写
过的研究成果,不包含本人或其他已申请学位或其他用途使用过的成果。对本文
的研究做出重要贡献的个人和集体,均已在文中以明确方式表明。

本人学位论文与资料若有不实,愿意承担一切相关的法律责任。

\begin{flushright}

~~~~~~~~~~~~~~~~~~~~~~~~~~~~~学位论文作者签名:~~~~~~~~~~~~~~~~~~~~~~~~~~~~~~\\          
~~~~~~~~~~~~~~~~~~~~~~~~~~~~~~~~~~~~~~~~~年~~~~~~月~~~~~~日~~~~~~~~~~
\end{flushright}

   % 承诺

\clearpage
\ifx\atempxetex\usewhat\else
\end{CJK*}
\fi

\end{document}

%%% Local Variables: 
%%% mode: latex
%%% TeX-master: t
%%% End: 
